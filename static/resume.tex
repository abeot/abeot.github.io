\documentclass{denniscv}

\begin{document}

\name{Albert Cao}

\sociallist{
    \social{\Letter}{\href{mailto:albertcao100@gmail.com}{albertcao100@gmail.com}}\sep%
    \social{\faGithub}{\href{https://github.com/abeot}{abeot}}\sep%
    \social{\faLinkedin}{\href{https://www.linkedin.com/in/abeot/}{abeot}}
}

\cat{Education}

\act{Centennial High School}{Fall 2022}{present}

\desc{Graduating in 2026. Current unweighted GPA is 4.0, weighted 4.93.}

\act{Burleigh Manor Middle School (BMMS)}{Fall 2019}{Summer 2022}

\desc{Unweighted GPA 4.0, 1st in grade. Was part of the county's accelerated math program, took AP Statstics, scored a 5.}

\cat{Executive Summary}

\desc{
    \begin{itemize}
    \item Passionate about problem solving, and loves to explore mathematics and computer programming
    \item In-depth understanding about various fields of mathematics, including but not limited to precalculus, calculus, statistics, number theory, and probability
    \item Strong leadership evidenced by successful tutoring of elementary school students in competitive math as well as computer programming
    \item Strong and self-driven initiator shown by taking Coursera courses from top universities including Princeton and Stanford on various computer science topics
    \item Strong communication and interpersonal skills through being on the student government association and organizing opportunities and events for students
\end{itemize}
}

\cat{Internship and Volunteer Experience}

\act{Intern at University of Maryland, School of Pharmacy}{2022}{present}

\desc{Researched the application of supervised machine learning in modeling the mortality of Busulfan in Hematopoietic Stem Cell Transplantation, for identifying risk factors.}
\refer{Reference: Prof. Joga Gobburu.}

\act{BMMS Student Government Representative}{Fall 2021}{Summer 2022}

\desc{Was part of the Student Government Association (SGA). Organized fundraising and community events to donate to local charities and homeless shelters. Also addressed school policies and student well being.}
\refer{Reference: Kelsea Valance.}

\act{Acton Institute of Computer Science}{Fall 2020}{Winter 2021}

\desc{Helped teach classes, grade homework, and give feedback.}
\refer{Reference: Charles Wang}

\act{Local Elementary School Tutoring}{Summer 2021}{}

\desc{Gave individual private lessons to elementary schoolers on middle school math topics. Also taught problem solving strategies necessary for introductory competition math.}

\cat{Accepted Publications}

\desc{\newline Allison Dunn, \textbf{Albert Cao}, Joga Gobburu, Janel Long-Boyle, Rahul Goyal. Modeling Mortality Of Pediatric Patients Undergoing Hematopoietic Stem Cell Transplantation Using Supervised Machine Learning, American Society for Clinical Pharmacology and Therapeutics Annual Meeting, March 2023, Atlanta, Georgia. }

\desc{\newline
\indent Currently, there is an unmet need for the identification of factors that may increase mortality in pediatric patients undergoing hematopoietic stem cell transplantation (HSCT). A machine learning (ML) method for risk factor discovery by modeling mortality was developed for this purpose.

Patient-related, treatment-related, and survival outcome data was retrospectively collected from pediatric patients who underwent HSCT at the University of California San Francisco Medical Center and analyzed using Pumas v2.1. Due to low sample size and class imbalance, a total of 200 training and testing samples were prepared using stratified bootstrapping to predict 1-year mortality. Ten ML algorithms were trained on the samples. The final model was selected based on sensitivity, specificity, and accuracy, and feature importance was obtained using the mean absolute Shapley value for each feature across all samples.

Of the 68 subjects analyzed 25\% were decedents. Gaussian Naïve Bayes algorithm demonstrated optimal performance, with a median sensitivity of 83\%, specificity of 32\%, and accuracy of 43\%. The risk factors identified as of highest importance included donor relation, degree of mismatch, serotherapy regimen, requiring retransplant, and days to absolute neutrophil count exceeding 500 cells/µL.

To the best of our knowledge, this is the first time a ML approach was used to predict 1-year mortality and identify risk factors for pediatric patients undergoing HSCT. Due to the low sample size, a median sensitivity of 83\% justified the adequacy in model performance. This model suggests that receiving cells from an unrelated donor and the degree of donor cell mismatch are the greatest risk factors for mortality.}

\cat{Awards}

\act{Honor Roll of Distinction (Top 1\%) for the American Mathematics Competitions (AMC) 10}{2022}{}

\act{American Invitational Mathematics Examination qualifier}{2021,2022}{}

\act{USA Computing Olympiad Silver Division}{2022}{}

% \cat{Projects}

\cat{Skills}

\act{Python, Java, C++}{}{}

\desc{Experience in data structures and algorithms using Python, Java, and C++ for competitive programming}

\act{Julia, Python with ML}{}{}

\desc{Experience with machine learning models and libraries using Julia and/or Python.}

\act{Java and Kotlin with Android}{}{}

\desc{Used Java and/or Kotlin to design Android applications.}

\act{LaTeX}{}{}

\desc{Used LaTeX to typeset and design documents}

\cat{Other Experiences and Activites}

\act{Fencing}{DC Fencers Club, 2017}{present}

\desc{A2022-rated fencer, nationally ranked in Cadet Men's Epee.}

\act{Cello}{Dr. Maxim Kozlov, 2017}{2022}

\desc{Was in the Maryland Junior All State Orchestra, 
selected to the Howard County High School Gifted and Talented Orchestra (HSGTO)
as well as the Middle School Gifted and Talented Orchestra (MSGTO).
First chair of the school's string orchestra.}

\end{document}

